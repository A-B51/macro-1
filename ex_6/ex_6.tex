\documentclass[11pt]{article} % font size
\usepackage{graphicx}
\usepackage[usenames, dvipsnames]{color}
\usepackage{url}
\usepackage{bbm}
\usepackage{setspace}
\usepackage{amssymb}
\usepackage{amsfonts}
\usepackage{amsmath}
\usepackage{amsthm}
\usepackage{rotating}
\usepackage{harvard}
\usepackage{enumitem}
\usepackage{filecontents}
\usepackage{verbatim}
\usepackage{subfiles}
\usepackage{csvsimple}
\usepackage{pdflscape}
\usepackage{longtable}
\usepackage{listings}
\usepackage[utf8]{inputenc} % package for nonstandard letters (umlauts etc)
\usepackage{natbib} % bibliography
\usepackage{booktabs} % For nice tables
% setting up the design
\usepackage[paper=a4paper,left=30mm,right=30mm,top=30mm,bottom=30mm]{geometry}
\renewcommand{\baselinestretch}{1.5} % line spacing
\renewcommand{\topfraction}{0.99}
\renewcommand{\bottomfraction}{0.99}
\renewcommand{\textfraction}{0.01}
\renewcommand{\floatpagefraction}{0.99}
\setlength{\footnotesep}{5mm}
\setlength{\parindent}{0em} % length of indent
\setlength{\parskip}{1em} % length of paragraph skip
\sloppy % to avoid messy line breaks
% some shortcuts and user-declared macros
\def\k{\ensuremath\kappa}
\def\l{\ensuremath\lambda}
%\addtolength{\oddsidemargin}{-0.5cm} % paper dimensions
%\addtolength{\evensidemargin}{-0.5cm}
%\addtolength{\textwidth}{1cm}
%\addtolength{\topmargin}{-0.5cm}


\begin{document}

\thispagestyle{empty}
\ \vspace{1.0cm}
\begin{center}
{\LARGE
MACRO III \\
\textit{Exercise 6} \\
{\small - Structural Unemployment -}\\[2cm]
}

{
Major in Economics \\
{University of St. Gallen} \\ [2cm]
Instructor:\\
Cozzi Guido \\[2cm]
}

{
Group:\\
Amiet Pascal (18-605-428)\\
Bittencourt André (17-622-887)\\
LeRoy Juliette (18-614-008)\\
}
\end{center}

\pagebreak

\textbf{\Large{Exercise 6.1}}

\textit{In the figure below, downloaded from the U.S. Bureau of Labor Statistics,
https://www.bls.gov/charts/employment-situation/civilian-unemployment-rate.htm#,
you can observe the evolution of the monthly total unemployment rate over
the past 20 years.}\\

\textit{a) How is the rate of unemployment defned and measured by the U.S.
Bureau of Labor Statistics? In particular, how is the number of un-
employed counted? How is the labor force measured?}




\pagebreak
\textit{b) Explain in one paragraph the evolution of the unemployment rate over
this period.}\par 





\pagebreak
\textit{c) Download the unemployment rate series from https://data.bls.gov/ for
the sample period 1948-2020. Plot the series. Compute the average
unemployment rate over this period and brie
y discuss the value.}\par 




\pagebreak
\textit{d) (OPTIONAL) Discuss the effects of the coronavirus (COVID-19) pan-
demic and efforts to contain it. Make a comparison with the unem-
ployment rate in another country of your choice. Note that you should
report the data source.}\par




\pagebreak
\textit{e) (OPTIONAL) Do unemployment rates vary by gender, ethnicity, or
educational attainment?}\par




\pagebreak
\textbf{\Large{Exercise 6.2}}

\textit{Consider an efficiency wage model with one firm, as discussed in chapter 11.2
of the textbook. The firm's profits are $zR\left(a(w')L\right)-w'L$, where $w'$ denotes the real wage paid by this firm. Furthermore, assume that the efficiency function is}
\begin{equation}
    a=(w'-v)^{\eta}, \;\;\;\;\; 0\leq\eta<1
\end{equation}
\textit{where $v$ is the employee's outside option.}

\textit{a) Find the first order conditions for the firm's profit maximization problem with respect to $w'$ and $L$. Derive the Solow condition by combining the two conditions. Find the expression for the optimal wage as a function of $v$ and $\eta$.}\par




\pagebreak
\textit{b) How does the optimal wage depend on $v$ and $\eta$? Explain the economic
mechanisms.}\par




\pagebreak
\textit{From now on, assume that the outside option is}
\begin{equation}
    v = (1-u)w+ucw, \;\;\;\;\; 0<c<1
\end{equation}
\textit{where $u$ is the unemployment rate, $c$ is the replacement ratio, and $w$ is the general real wage level.}


\bigskip
\textit{c) Using the outside option, express the optimal wage rate of the representative firm as a function of $\eta$, $u$, $w$ and $c$. Then use the equilibrium
condition $w=w'$ to derive the equilibrium rate of unemployment $u^*$.} \par



\pagebreak
\textit{d) How do the parameters $\eta$ and $c$ affect $u^*$? Explain the economic mechanisms.}




\pagebreak
\textit{e) Insert the expression you found for $u^*$ into the outside option $v$; this
gives you $v^*$. The plug this $v^*$ into the efficiency function $a$. Assuming
$\eta=0.01$, what is the minimum the firm has to pay if it wants its
workers to exert any effort?}




\pagebreak
\textit{f) (OPTIONAL) Now assume that a proportional labor income tax is
introduced. The tax is levied on the workers, not on the firms. How
does this impact the outside option equation? What are the effects on
equilibrium unemployment $u^*$?}




\pagebreak
\textit{g) (OPTIONAL) How does your answer to f) change if the unemployment
benefits are taxed at the same rate?}

\end{document}
